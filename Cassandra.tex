\section{Кратко о Cassandra}

Apache Cassandra ~--- распределенная система управления базами данных, относящаяся к классу NoSQL. За счет отказа от реляционности и транзакционности в NoSQL системах достигаются возможности хорошей горизонтальной масштабируемости и репликации. Это направление в компьютерных науках сейчас находится в стадии активного развития, и практически у каждой крупной IT-компании есть своя NoSQL база данных. Cassandra изначально была разработкой Facebook, однако в 2009 году было решено отдать проект фонду Apache Software.

В первом приближении на Cassandra можно смотреть как на следующие сущности (в порядке вложенности):

\begin{enumerate}
	\item Кластер ~--- множество серверов, на которых хранится множество баз данных;
	\item Пространство ключей ~--- база данных, множество таблиц;
	\item Семейство колонок ~--- таблица, множество элементов;
	\item Колонка ~--- ячейка, хранящая в себе конкретную запись.
\end{enumerate}

Колонка содержит в себе следующую информацию:

\begin{enumerate}
	\item Имя строки, в которой лежит ячейка;
	\item Имя колонки, в которой лежит ячейка;
	\item Набор байтов с хранимой информацией;
	\item Время создания ячейки;
	\item Время жизни ячейки.
\end{enumerate}

Фактически семейство колонок ~--- разреженная таблица, в которой каждая строка содержит множество ячеек, упорядоченное по имени колонки в лексикографическом порядке. Порядок колонок в строке ~--- важная особенность хранения данных, она является ключевой для предлагаемого алгоритма.