\section{Apache Cassandra}

Apache Cassandra~--- распределенная система управления базами данных, относящаяся к классу NoSQL. За счет отказа от реляционности и транзакционности в NoSQL системах достигаются возможности хорошей горизонтальной масштабируемости и репликации. Это направление в компьютерных науках сейчас находится в стадии активного развития, и очень многие компании стали использовать такие базы для решения большого числа задач. Cassandra изначально была разработкой Facebook, однако в 2009 году было решено отдать проект фонду Apache Software.

В первом приближении на Cassandra можно смотреть как на следующие сущности (в порядке вложенности):

\begin{enumerate}
	\item кластер~--- множество серверов, на которых хранится множество баз данных;
	\item пространство ключей~--- база данных, множество таблиц;
	\item семейство колонок~--- таблица, множество элементов;
	\item колонка~--- ячейка, хранящая в себе конкретную запись.
\end{enumerate}

Колонка содержит в себе следующую информацию:

\begin{enumerate}
	\item имя строки, в которой лежит ячейка;
	\item имя колонки, в которой лежит ячейка;
	\item набор байтов с хранимой информацией;
	\item временная отметка ячейки;
	\item время жизни ячейки.
\end{enumerate}

Фактически семейство колонок~--- разреженная таблица, в которой каждая строка содержит множество ячеек, упорядоченных по имени колонки в лексикографическом порядке. Порядок колонок в строке~--- важная особенность хранения данных, она является ключевой для предлагаемого алгоритма.

Временная отметка ячейки используется следующим образом: если поступит запрос на запись ячейки A, а в таблице уже существует ячейка B с такими же координатами, то произойдет сравнение временных отметок ячеек А и В, и ячейка А будет записана только в том случае, если ее отметка больше чем отметка ячейки В.

Cassandra предоставляет множество возможностей для записи и чтения данных. Среди них стоит отметить следующие:

\begin{itemize}
	\item записать одну ячейку;
	\item записать множество ячеек в одно семейство колонок;
	\item вычитать последовательно из строки заданное количество ячеек начиная с ячейки с заданным именем колонки.
\end{itemize}

Существует утверждение, известное как CAP теорема. Суть его в том, что при реализации распределенных вычислений возможно обеспечить не более двух из трех следующих свойств:

\begin{itemize}
	\item согласованность (consistency)~--- во всех вычислительных узлах в один момент времени данные не противоречат друг другу;
	\item доступность (availability)~--- любой запрос к распределенной системе завершается корректным откликом;
	\item устойчивость к разделению (partition tolerance)~--- расщепление распределенной системы на несколько изолированных секций не приводит к некорректности отклика от каждой из секций.
\end{itemize}

При конфигурировании Cassandra есть возможность выбрать между согласованностью и доступностью, мы выбрали согласованность. В частности, это означает, что после записи ячейки в некоторую строку результаты запросов на чтение этой строки обязательно будут содержать эту ячейку. Этот факт будет использован далее при доказательстве корректности алгоритмов.
