\section{Apache Cassandra}

$Apache\;Cassandra$~--- распределенная система управления базами данных, относящаяся к классу NoSQL. За счет отказа от реляционности и транзакционности в NoSQL системах достигаются возможности хорошей горизонтальной масштабируемости и репликации. Это направление в компьютерных науках сейчас находится в стадии активного развития, и очень многие компании стали использовать такие базы для решения большого числа задач. Cassandra изначально была разработкой Facebook, однако в 2009 году было решено отдать проект фонду Apache Software.

В первом приближении на Cassandra можно смотреть как на следующие сущности (в порядке вложенности):

\begin{itemize}
	\item кластер~--- множество серверов, на которых хранится множество баз данных;
	\item пространство ключей~--- база данных, множество таблиц;
	\item семейство колонок~--- таблица, множество элементов;
	\item колонка~--- ячейка, хранящая в себе конкретную запись.
\end{itemize}

Ячейка содержит в себе следующую информацию:

\begin{itemize}
	\item имя строки, в которой лежит ячейка;
	\item имя колонки, в которой лежит ячейка;
	\item набор байтов с хранимой информацией;
	\item временная отметка ячейки;
	\item время жизни ячейки.
\end{itemize}

Фактически семейство колонок является разреженной таблицей, в которой каждая строка содержит множество ячеек, упорядоченных по имени колонки в лексикографическом порядке. Порядок ячеек в строке~--- важная особенность хранения данных, являющаяся ключевой для предлагаемого алгоритма.

Cassandra предоставляет множество возможностей для записи и чтения данных. Среди них стоит отметить следующие:

\begin{itemize}
	\item записать одну ячейку;
	\item удалить ячейку с заданными координатами;
	\item записать множество ячеек в одно семейство колонок;
	\item вычитать последовательно из строки заданное количество ячеек начиная с ячейки с заданным именем колонки.
\end{itemize}

При удалении ячейки из строки на ее место записывается специальная ячейка, называемая маркером удаления, которая означает, что в данной строке и данной колонке ячейка когда-то существовала, но была удалена.

При синхронизации данных между узлами возникает потребность в выборе лучшей ячейки из нескольких с одинаковыми координатами. В этом случае ячейки сравниваются с помощью следующего алгоритма:
\begin{lstlisting}[language=csh,caption={Алгоритм GetBestCell(cellA, cellB)}]
  1.  |Если ячейки cellA и cellB имеют разные временные отметки:|
  2.  	|Вернуть ячейку с большей временной отметкой|
  3.  |Иначе:|
  4.  	|Если одна из ячеек является маркером удаления:|
  5.  		|Вернуть ячейку, являющуюся маркером удаления|
  6.  	|Иначе:|
  7.  		|Посчитать MD5 от набора байтов с хранимой информацией|
  8.  		|Вернуть ячейку, MD5 которой меньше|
\end{lstlisting}

Кластер Cassandra состоит из множества узлов. Каждая строка из семейства колонок хранится не на всех узлах, а только на $ReplicationFactor$ из них, $ReplicationFactor$ указывается при конфигурировании базы. Выполнение запроса на чтение или запись данных в строке возможно на любом узле, при этом он будет выполнять роль координатора запроса и взаимодействовать с узлами, отвечающими за хранение этой строки.

Cassandra поддерживает разные стратегии чтения и записи, отличающиеся друг от друга количеством узлов, от которых будет ожидаться ответ. Мы используем стратегии $WriteQuorum$ для записи и $ReadQuorum$ для чтения. При использовании стратегии $WriteQuorum$ в момент записи данных в конкретную строку координатор пошлет запрос на все узлы, отвечающие за хранение данной строки и дождется ответа хотя бы от $\lfloor \frac{ReplicationFactor}{2} + 1\rfloor$ из них. Каждый узел, получивший запрос на запись, попытается обновить данные в ячейке с использованием алгоритма $GetBestCell$.
Аналогично, в момент чтения координатор запроса дождется ответа хотя бы от $\lfloor \frac{ReplicationFactor}{2} + 1\rfloor$ узла и выберет лучшую ячейку с использованием алгоритма $GetBestCell$.

Существует утверждение, известное как \textit{теорема CAP}. Суть его в том, что при реализации распределенных вычислений возможно обеспечить не более двух из трех следующих свойств:

\begin{itemize}
	\item согласованность (consistency)~--- в любой момент времени один и тот же запрос к любому узлу в случае успеха даст один и тот же ответ;
	\item доступность (availability)~--- любой запрос к распределенной системе завершается корректным откликом;
	\item устойчивость к разделению (partition tolerance)~--- расщепление распределенной системы на несколько изолированных секций не приводит к некорректности отклика от каждой из секций.
\end{itemize}

Cassandra гарантирует, что использование стратегий $ReadQuorum$ и \\$WriteQuorum$ дает согласованность и устойчивость к разделению. В частности, это означает, что после записи ячейки в некоторую строку результаты запросов на чтение этой строки обязательно будут содержать эту ячейку. Этот факт будет использован далее при доказательстве корректности алгоритмов.